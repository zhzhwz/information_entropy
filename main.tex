\documentclass{beamer}
\usetheme{Madrid}

\usepackage{ctex}

\newcommand{\concept}[1]{{\color{blue} #1}}

\title{信息熵与数据压缩}
\author{祝润天}
\institute{复旦大学计算机科学技术学院}
\date{2024 年 10 月 11 日}

\begin{document}

\begin{frame}
    
    \maketitle

\end{frame}

\begin{frame}
    \frametitle{信息的度量}

    \begin{block}{抛硬币}
        抛一枚硬币,平均意义下,最少需要几个比特来表示得到的结果?
    \end{block}

    抛出反面记为 $0$,抛出正面记为 $1$。则两种情况均需要 $1$ 个比特来表示结果。平均需要

    \[E = \frac{1}{2}\times 1 + \frac{1}{2}\times 1 = 1\]

    个比特。

\end{frame}

\begin{frame}
    \frametitle{信息的度量}

    \begin{block}{投骰子}
        投一个骰子,平均意义下,最少需要几个比特来表示得到的结果?
    \end{block}
    
    将 $6$ 种结果分别编码为

    \[\begin{split}
        & 1 \rightarrow 000, 2 \rightarrow 001, 3 \rightarrow 010 \\
        & 4 \rightarrow 011, 5 \rightarrow 100, 6 \rightarrow 101
    \end{split}\]

    平均需要 $3$ 个比特。

    能否更少?

\end{frame}

\begin{frame}
    \frametitle{信息的度量}

    \begin{block}{投骰子}
        投一个骰子,平均意义下,最少需要几个比特来表示得到的结果?
    \end{block}
    
    将 $6$ 种结果分别编码为

    \[\begin{split}
        & 1 \rightarrow 000, 2 \rightarrow 001, 3 \rightarrow 010 \\
        & 4 \rightarrow 011, 5 \rightarrow 10, \;\; 6 \rightarrow 11
    \end{split}\]

    平均需要

    \[E = \frac{4}{6}\times 3 + \frac{2}{6}\times 2 \approx 2.67\]
    
    个比特。

\end{frame}

\begin{frame}
    \frametitle{文件数据压缩}

    \begin{example}
        你说的对,但是《原神》是由米哈游自主研发的一款全新开放世界冒险游戏。游戏发生在一个被称作「提瓦特」的幻想世界,在这里,被神选中的人将被授予「神之眼」,导引元素之力。你将扮演一位名为「旅行者」的神秘角色,在自由的旅行中邂逅性格各异、能力独特的同伴们,和他们一起击败强敌,找回失散的亲人。
    \end{example}

    \begin{block}{汉字字频}
        \begin{table}
            \begin{tabular}{cccc}
                的 & 一 & 是 & $\cdots$ \\
                $0.0575$ & $0.0473$ & $0.0429$ & $\cdots$
            \end{tabular}
        \end{table}
    \end{block}

    使用等长的 Unicode 码 $\rightarrow$ 为概率更高的字符分配更短的码字

\end{frame}

\begin{frame}
    \frametitle{唯一可译码}

    将一个文件编码后,能够确保恢复这个文件。

    % \begin{block}{唯一可译码}
    %     令 $\mathcal{X}$ 是一个有限集合。从 $\mathcal{X}$ 中取出有限个元素组成一个序列,则称这个序列为一个 $\mathcal{X}$ 上的\concept{字符串}。把全体 $\mathcal{X}$ 上的字符串记为 $\mathcal{X}^*$。给定两个有限集合 $\mathcal{X}$ 和 $\mathcal{Y}$,函数 $c: \mathcal{X}\rightarrow \mathcal{Y}^*$ 称为一个\concept{字符编码}。令 $x_i\in \mathcal{X}$, $c^*(x_1, \dots, x_n) = c(x_1)\cdots c(x_n)$。如果 $c^*$ 是一个单射,则称 $c$ 是一个唯一可译码。
    % \end{block}

    \begin{example}
        $c(x)$ 将 $\mathcal{X} = \{1, 2, 3, 4, 5, 6\}$ 中的每个字符 $x$ 编码为它的二进制表示,对应的码字分别为 $1, 10, 11, 100, 101, 110$。则 $c$ 不是一个唯一可译码,因为 $110$ 可以被译成 $6$ 或者 $12$。
    \end{example}

    \begin{itemize}
        \item 令 $c^*(x_1, \dots, x_n) = c(x_1)\cdots c(x_n)$。则 $c$ 是一个唯一可译码当且仅当 $c^*$ 是一个单射。
    \end{itemize}

\end{frame}

\begin{frame}
    \frametitle{信息熵}

    \begin{block}{信息熵}
        假设 $X$ 是一个离散型随机变量,其可能的取值集合为$\mathcal{X}$。则 $X$ 的信息熵定义为

        \[H(X) = -\sum_{x\in \mathcal{X}}p(x)\log p(x)\]
    \end{block}

    \begin{block}{符号码信源编码定理}
        若编码 $c$ 将 $x$ 编码为有限长的 $01$ 字符串 $c(x)$,则

        \[H(X) \leq E(\lvert c(X)\rvert)\]
    \end{block}

    % 假设 $n\in\mathbb{N}$, $X$ 的取值集合为 $\mathcal{X} = \{1, \dots, 2^n\}$,取到每个值的概率均为 $\frac{1}{2^n}$。最优的编码方案就是用 $\underbrace{0 \cdots 0}_{n}, \dots, \underbrace{1 \cdots 1}_{n}$ 编码每个取值。平均需要的比特数为

    % \[\sum_{x\in\mathcal{X}}\frac{1}{2^n}\log 2^n = -\sum_{x\in\mathcal{X}}\frac{1}{2^n}\log \frac{1}{2^n} = H(X)\]

    % 更一般地,对任意离散型随机变量 $X$,要表示 $X$ 取值的结果,平均需要的比特数不会少于 $H(X)$。

\end{frame}

\begin{frame}
    \frametitle{Kraft - McMillan 不等式}

    对 $\mathcal{X}$ 中的字符 $x$,$c$ 将其编码为对应的 $01$ 字符串 $c(x)$。令 $l_x = \lvert c(x)\rvert$ 是码字的长度。则 $\sum_{x\in\mathcal{X}}2^{-l_x}\leq 1$。

    \begin{proof}
        令 $l_{min} = \min_{x\in\mathcal{X}}\lvert c(x)\rvert$, $l_{max} = \max_{x\in\mathcal{X}}\lvert c(x)\rvert$,$a(k)$ 为长度为 $k$ 的码字个数。则
        \[\left(\sum_{x\in\mathcal{X}}2^{-\lvert c(x)\rvert}\right)^n = \sum_{k = nl_{min}}^{nl_{max}} a(k)2^{-k}\]
        由唯一可译性质码的性质可知 $a(k)\leq 2^k$,则
        \[\sum_{x\in\mathcal{X}}2^{-\lvert c(x)\rvert} \leq (n(l_{max} - l_{min} + 1))^{\frac{1}{n}}\]
        令 $n\to +\infty$ 则得到结论。
    \end{proof}

\end{frame}

\begin{frame}
    \frametitle{符号码信源编码定理}

        \[\begin{split}
            minimize & \quad \sum_{x\in\mathcal{X}}p(x)l_x \\
            s.t. & \quad \sum_{x\in\mathcal{X}} 2^{-l_x} \leq 1 \text{ and } l_x > 0
        \end{split}\]

    \begin{proof}
        极值只会在 $\sum_{x\in\mathcal{X}} 2^{-l_x} = 1$ 时取到。利用拉格朗日乘数法,构造

        \[L(l_{x_1}, \dots, l_{x_n}, \lambda) = \sum_{x\in\mathcal{X}} p(x)l_x - \lambda\left(\sum_{x\in\mathcal{X}}2^{-l_x}-1\right)\]
        
        解得

        \[l_x = -\log p(x)\]

    \end{proof}

\end{frame}

\begin{frame}
    \frametitle{符号码信源编码定理}
    
    因此,当 $l_x = -\log p(x)$ 时,$\sum_{x\in\mathcal{X}}p(x)l_x$ 取到极小值

    \[-\sum_{x\in\mathcal{X}}p(x)\log p(x) = H(X)\]

    从而

    \[H(X) \leq \sum_{x\in\mathcal{X}}p(x)l_x = E(\lvert c(X)\rvert)\]

    这就是符号码信源编码定理。

\end{frame}

\begin{frame}
    \frametitle{信源编码定理(香农第一定理)}

    \begin{block}{信源编码定理}
        设离散型随机变量 $X$ 的取值集合为 $\mathcal{X}$。对任意 $\varepsilon > 0$,存在一个整数 $n$ 和一个唯一可译码 $c: \mathcal{X}^n \rightarrow \{0, 1\}^*$,使得

        \[\frac{1}{n}E(\lvert c(X_1, \dots, X_n)\rvert) \leq H(X) + \varepsilon\]
    \end{block}

    \begin{itemize}
        \item 弱典型序列:数量少,但出现概率大
        \item 给弱典型序列分配短编码,给其他序列分配长编码
    \end{itemize}

\end{frame}

\begin{frame}
    \frametitle{典型序列}

    假设 $X_1, \dots, X_n$ 为取值为 $\{0, 1\}$,独立同分布的随机变量,取到 $0$ 的概率为 $p$,$q = p - 1$。则其中最``典型''的序列应该有 $p$ 个 $0$ 和 $q$ 个 $1$,例如 $\underbrace{0\cdots 0}_{p} \underbrace{1\cdots 1}_{q}$。某个特定的``典型序列'' $(x_1, \dots, x_n)$ 出现的概率为

    \[P((X_1, \dots, X_n) = (x_1, \dots, x_n)) = p^{np} q^{nq}\]

    两边取 $\log$,得到

    \[\log P((X_1, \dots, X_n) = (x_1, \dots, x_n)) = np\log p + nq\log q = -nH(X)\]

    即

    \[P((X_1, \dots, X_n) = (x_1, \dots, x_n)) = 2^{-nH(X)}\]

    则典型序列的数量不会超过 $2^{nH(X)}$。

\end{frame}

\begin{frame}
    \frametitle{弱典型序列}
    
    \begin{block}{弱典型序列}
        对任意 $n\in \mathbb{N}$ 和 $\varepsilon > 0$,令

        \[\mathcal{T}_n^{\varepsilon} = \left\{(x_1, \dots, x_n) \in \mathcal{X}^n: \left\lvert-\frac{1}{n}\log p_{X_1, \dots, X_n}(x_1, \dots, x_n) - H(X)\right\rvert \leq \varepsilon\right\}\]

        将 $\mathcal{T}_n^{\varepsilon}$ 中的序列称作长度为 $n$,误差为 $\varepsilon$ 的弱典型序列。
    \end{block}

\end{frame}

\begin{frame}
    \frametitle{弱典型序列是``少''的}
    
        对任意的弱典型序列,有 $\left\lvert-\frac{1}{n}\log p_{X_1, \dots, X_n}(x_1, \dots, x_n) - H(X)\right\rvert \leq \varepsilon$,即
        \[p_{X_1, \dots, X_n}(x_1, \dots, x_n) \in \left[2^{-n(H(X)+\varepsilon)}, 2^{-n(H(X)-\varepsilon)}\right]\]
        因此

        \[\begin{split}
            1 = \sum_{(x_1, \dots, x_n)\in\mathcal{X}}p_{X_1, \dots, X_n}(x_1, \dots, x_n) \geq & \sum_{(x_1, \dots, x_n)\in\mathcal{T}_n^{\varepsilon}}p_{X_1, \dots, X_n}(x_1, \dots, x_n) \\
            \geq & \sum_{(x_1, \dots, x_n)\in\mathcal{T}_n^{\varepsilon}} 2^{-n(H(X)+\varepsilon)}
        \end{split}\]
        从而
        \[\left\lvert\mathcal{T}_n^{\varepsilon}\right\rvert \leq 2^{n(H(X) + \varepsilon)}\]

\end{frame}

\begin{frame}
    \frametitle{弱典型序列的出现概率很大}

    \begin{block}{弱大数定律(辛钦定理)}
        设 $X_1, X_2, \dots$ 独立同分布且期望存在,则对任意 $\varepsilon > 0$,有

        \[\lim_{n\to +\infty} P\left(\left\lvert \frac{1}{n}\sum_{i = 1}^n X - E(X)\right\rvert < \varepsilon\right) = 1\]
    \end{block}

    对任意序列,$X_1, \dots, X_n$ 独立同分布,则

    \[-\log p_{X_1, \dots, X_n}(X_1, \dots, X_n) =  -\sum_{i=1}^{n} \log p_X(X_i)\]

    且 $E(-\log p_X(X_i)) = H(X)$。则

    \[\lim_{n\to +\infty} P\left(\left\lvert -\frac{1}{n}\sum_{i=1}^{n} \log p_X(X_i) - H(X)\right\rvert < \varepsilon\right) = 1\]

\end{frame}

\begin{frame}
    \frametitle{弱典型序列的出现概率很大}

    任取 $\varepsilon_0$,对于足够大的 $n$,有

    \[P\left(\left\lvert -\frac{1}{n}\sum_{i=1}^{n} \log p_X(X_i) - H(X)\right\rvert < \varepsilon\right) \geq 1 - \varepsilon_0\]

    取 $\varepsilon_0 = \varepsilon$,则

    \[\begin{split}
        P((X_1, \dots, X_n)\in \mathcal{T}_n^{\varepsilon}) = & P\left(\left\lvert-\frac{1}{n}\log p_{X_1, \dots, X_n}(X_1, \dots, X_n) - H(X)\right\rvert \leq \varepsilon\right) \\
        \geq & P\left(\left\lvert -\frac{1}{n}\sum_{i=1}^{n} \log p_X(X_i) - H(X)\right\rvert < \varepsilon\right) \\
        \geq & 1 - \varepsilon
    \end{split}\]

\end{frame}

\begin{frame}
    \frametitle{信源编码定理}

    \begin{block}{信源编码定理}
        设离散型随机变量 $X$ 的取值集合为 $\mathcal{X}$。对任意 $\varepsilon > 0$,存在一个整数 $n$ 和一个唯一可译码 $c: \mathcal{X}^n \rightarrow \{0, 1\}^*$,使得

        \[\frac{1}{n}E(\lvert c(X_1, \dots, X_n)\rvert) \leq H(X) + \varepsilon\]
    \end{block}
    
    将所有 $x\in \mathcal{X}$ 分为典型序列 $\mathcal{T}_n^{\varepsilon_0}$ 和非典型序列。

    \begin{itemize}
        \item 典型序列至多有 $2^{n(H(X) + \varepsilon_0)}$ 个,因此可以用 $l_1 = \lceil n(H(X) + \varepsilon_0)\rceil$ 个比特来表示;
        \item 非典型序列至多有 $\lvert \mathcal{X}\rvert$ 个,因此可以用 $l_2 = \lceil n\log \lvert \mathcal{X}\rvert\rceil$ 个比特来表示;
        \item 在典型序列的码字前加上 $0$,非典型序列的码字前加上 $1$,用于区分。
    \end{itemize}

\end{frame}

\begin{frame}
    \frametitle{信源编码定理}

    \[\begin{split}
        & E(\lvert c(X_1, \dots, X_n)\rvert) \\
        = & \sum_{x\in\mathcal{T}_n^{\varepsilon_0}}p(x)(l_1 + 1) + \sum_{x\notin\mathcal{T}_n^{\varepsilon_0}}p(x)(l_2 + 1) \\
        \leq & \sum_{x\in\mathcal{T}_n^{\varepsilon_0}}p(x)(n(H(X) + \varepsilon_0) + 2) + \sum_{x\notin\mathcal{T}_n^{\varepsilon_0}}p(x)(n\log \lvert X\rvert + 2) \\
        = & P((X_1, \dots, X_n) \in \mathcal{T}_n^{\varepsilon_0})(n(H(X) + \varepsilon_0) + 2) \\
        & + P((X_1, \dots, X_n) \notin \mathcal{T}_n^{\varepsilon_0})(n\log \lvert X\rvert + 2) \\
        \leq & n(H(X) + \varepsilon_0) + 2 + \varepsilon_0 n\log\lvert\mathcal{X}\rvert = n(H(X) + \epsilon_1)
    \end{split}\]

    其中 $\varepsilon_1 = \varepsilon_0(1 + \log\lvert\mathcal{X}\rvert) + \frac{2}{n}$。取足够小的 $\varepsilon_0$ 使得 $\varepsilon_0(1 + \log\lvert\mathcal{X}\rvert) < \varepsilon / 2$,然后取足够大的 $n$ 使得 $\frac{2}{n} < \varepsilon / 2$。则有 $E(\lvert c(X_1, \dots, X_n)\rvert) \leq n(H(X) + \epsilon)$,即

    \[\frac{1}{n}E(\lvert c(X_1, \dots, X_n)\rvert) \leq H(X) + \varepsilon\]

\end{frame}

\end{document}